% Abhidhammāvatāra Main Document
% Main document for Theravada Buddhist Abhidhamma literature translation and commentary
% MUST be compiled with LuaLaTeX (supports Burmese and Pali script)
%%%%%%%%%%%%%%%%%%%%%%%%%%%%%%%%%%%%%%%%%%%%%%%%%%%%%%%%%%%%%%%

% Version Information
% ======================================================================
% Version: 1.0.0
% Release Date: 2025-09-02
% Author: Venerable Gunodaya Bhikkhu
% Maintainer: Venerable Gunodaya Bhikkhu
% License: Dhamma Dana (Dhamma Gift)
%
% Version History:
% 1.0.0 (2025-09-02) - Initial version release
%   - Completed basic Abhidhamma terminology system
%   - Implemented multilingual typesetting (English-Pali-Burmese)
%   - Established complete annotation system (footnotes, endnotes, margin notes)
%   - Designed professional academic typesetting format
%
% Compilation Requirements:
% - Engine: LuaLaTeX (Required)
% - Fonts: Padauk (Burmese), Times New Roman (Latin)
% - Packages: See abhidhammavatara.cls documentation
%
% Project Structure:
% - text/cover.tex     : Cover design
% - text/000.tex       : Editorial conventions
% - text/01-24.tex     : Chapter contents
% - text/term.tex      : Term definitions
% - references_Manual.bib: Bibliography database
%
% Important Notes:
% 1. MUST be compiled with LuaLaTeX to support Burmese script rendering
% 2. Ensure Padauk font is installed on your system
% 3. Uses biblatex for bibliography, NOT BibTeX
% =====================================================================%

% Document Class Settings
\documentclass[]{Abhidhammāvatāra} % Use the custom class

% Bibliography Database Setup
\addbibresource{references_Manual.bib} % Load bibliography database for biblatex
% Note: This uses biblatex, NOT BibTeX

%%%%%%%%%%%%%%%%%%%%%%%%%%%%%%%%%%%%%%%%%%%%%%%%%%%%%%%%%%%%%%%

% Document Metadata
\title{\textbf{USER MANUAL}} % Pali title
\author{GUṆODAYA BHIKKHU}
\date{} % Publication date left empty

%%%%%%%%%%%%%%%%%%%%%%%%%%%%%%%%%%%%%%%%%%%%%%%%%%%%%%%%%%%%%%%

\begin{document}

% Cover Page
\include{text/cover} % Cover design file

% Front Matter (Roman numeral pagination)
\frontmatter

% Title Page
\maketitle % Print title page information
\cleardoublepage % Blank page

% Index Collection Setup (commented, enable as needed)
%\makeindex % Enable index collection

%%%%%%%%%%%%%%%%%%%%%%%%%%%%%%%%%%%%%%%%%%%%%%%%%%%%%%%%%%%%%%%

% Table of Contents System Setup

% Brief Table of Contents (Chapter level only)
\setcounter{tocdepth}{0} % Show only chapter level
\renewcommand{\contentsname}{Brief Contents} % Brief TOC title
\tableofcontents % Print brief table of contents

% Detailed Table of Contents (Full structure)
\renewcommand{\contentsname}{Detailed Contents} % Detailed TOC title

% Note: The following code is commented to allow deeper section levels
% \addtocontents{toc}{\protect\setcounter{tocdepth}{1}}

%%%%%%%%%%%%%%%%%%%%%%%%%%%%%%%%%%%%%%%%%%%%%%%%%%%%%%%%%%%%%%%

% Editorial Conventions Chapter
\chapter{How to Use This Typesetting Template}

\starthere

\section{Pali Text}
8.
Cittaṃ cetasikaṃ rūpaṃ, nibbānanti niruttaro;
Catudhā desayī dhamme, catusaccappakāsano.

Tattha cittanti visayavijānanaṃ cittaṃ, tassa pana ko vacanattho? Vuccate – sabbasaṅgāhakavasena pana cintetīti cittaṃ, attasantānaṃ vā cinotītipi cittaṃ.

9.
Vicittakaraṇā cittaṃ, attano cittatāya vā;
Paññattiyampi viññāṇe, vicitte cittakammake;
Cittasammuti daṭṭhabbā, viññāṇe idha viññunā.

Taṃ pana sārammaṇato ekavidhaṃ, savipākāvipākato duvidhaṃ. Tattha savipākaṃ nāma kusalākusalaṃ, avipākaṃ abyākataṃ. Kusalākusalābyākatajātibhedato tividhaṃ.

Tattha kusalanti panetassa ko vacanattho?

10.
Kucchitānaṃ salanato, kusānaṃ lavanena vā;
Kusena lātabbattā vā, kusalanti pavuccati.

\section{Burmese Translation}
 ၁။ အကျွန်ုပ်သည် ဤသို့ ကြားနာခဲ့ရပါသည် -
       အခါတစ်ပါး၌ မြတ်စွာဘုရားသည် များစွာသော ငါးရာခန့်မျှသော ရဟန်းသံဃာနှင့်အတူ ရာဇဂြိုဟ် (ပြည်) နှင့် နာဠန္ဒာ (မြို့) အကြား၌ ခရီးရှည်ကြွတော်မူ၏၊ သုပ္ပိယပရိဗိုဇ်သည်လည်း တပည့်ဖြစ်သော ဗြဟ္မဒတ်လုလင်နှင့်အတူ ရာဇဂြိုဟ် (ပြည်) နှင့် နာဠန္ဒာ (မြို့) အကြား၌ပင် ခရီးရှည်သွား၏။
       ထို (ခရီး) ၌ သုပ္ပိယပရိဗိုဇ်သည် ဘုရား တရား သံဃာ၏ အပြစ်ကို များစွာသော အကြောင်းဖြင့် ပြော၏၊ သုပ္ပိယပရိဗိုဇ်၏ တပည့်ဖြစ်သော ဗြဟ္မဒတ်လုလင်သည်ကား ဘုရား တရား သံဃာ၏ဂုဏ်ကို များစွာသော အကြောင်းဖြင့် ပြော၏၊ ဤသို့လျှင် ထိုဆရာတပည့်နှစ်ဦးတို့သည် အချင်းချင်း တိုက်ရိုက်ဆန့်ကျင်သော စကားကို ပြောလျက် မြတ်စွာဘုရားနှင့် ရဟန်းသံဃာ၏ နောက်မှ နောက်မှ အစဉ်လိုက်ကုန်၏။

\section{Academic Annotation System}
\canon{This text is from pāḷi canon.}

\commentary{This text is from aṭṭhakatha.}

\subcommentary{This text is from ṭīkā.}

\othertext{This text is from othertext.}

\section{Grammatical Analysis}
\subsection{ကိတ်}
\kitabasic{}{}{}

\kitabasic{ဗုဒ္ဓ}[မြတ်စွာဘုရား]{ဗုဓ}[သိ]{တ}[တတ်]

\kitafull{}{}{}{}

\kitafull{ဝိစာရ}{ဝိ}{စရ}{ဏ}

\kitafull*{}{}{}{}

\kitafull*{ဗုဒ္ဓ}{ဗုဓ}{ဏေ}[စေ]{တ}

\subsection{တဒ္ဓိတ်}
\taddhita{}{}{}

\subsection{သမာသ်}
\samasa{}{}{}

\subsection{Noun}
\namapada{buddho}{buddha}{si}

\subsection{Verb}
\akhyatapada{gacchati}{gamu}{a}{ti}


\section{Notes System}
\subsection{Inline Notes}
 ၁။ အကျွန်ုပ်သည် ဤသို့ ကြားနာခဲ့ရပါသည် -
       အခါတစ်ပါး၌ မြတ်စွာဘုရားသည်\inlinenote{This is Inline Notes from othertext.} များစွာသော ငါးရာခန့်မျှသော ရဟန်းသံဃာနှင့်အတူ\inlinenote*{This is Inline Notes from aṭṭhakatha.} ရာဇဂြိုဟ် (ပြည်) နှင့် နာဠန္ဒာ (မြို့) အကြား၌ ခရီးရှည်ကြွတော်မူ၏၊ သုပ္ပိယပရိဗိုဇ်သည်လည်း တပည့်ဖြစ်သော ဗြဟ္မဒတ်လုလင်နှင့်အတူ ရာဇဂြိုဟ် (ပြည်) နှင့် နာဠန္ဒာ (မြို့) အကြား၌ပင် ခရီးရှည်သွား၏။
       ထို (ခရီး) ၌ သုပ္ပိယပရိဗိုဇ်သည် ဘုရား တရား သံဃာ၏ အပြစ်ကို များစွာသော အကြောင်း\inlinenote*[]{This is Inline Notes from ṭīkā(ဋီကာ).}ဖြင့် ပြော၏၊ 

\subsection{End Notes}
သုပ္ပိယပရိဗိုဇ်၏ တပည့်ဖြစ်သော ဗြဟ္မဒတ်လုလင်သည်ကား ဘုရား တရား သံဃာ၏ဂုဏ်ကို\myendnote{အကျယ်ရှင်းလင်းချက်။ \commentary{This text is from aṭṭhakatha.}\footnotealt{\cite{atthasalini}}} များစွာသော အကြောင်းဖြင့် ပြော၏၊ ဤသို့လျှင် ထိုဆရာတပည့်နှစ်ဦးတို့သည် အချင်းချင်း တိုက်ရိုက်ဆန့်ကျင်သော\myendnote{another note.} စကားကို ပြောလျက် မြတ်စွာဘုရားနှင့် ရဟန်းသံဃာ၏ နောက်မှ နောက်မှ အစဉ်လိုက်ကုန်၏။

\subsection{Footnote}
Vicittakaraṇā cittaṃ, attano\footnotealt{This is footnote.} cittatāya vā;
Paññattiyampi viññāṇe, vicitte cittakammake;
Cittasammuti daṭṭhabbā, viññāṇe idha viññunā.


\subsection{Margin Notes}
သုပ္ပိယပရိဗိုဇ်၏ တပည့်ဖြစ်သော\marginnotealt{p.158} ဗြဟ္မဒတ်လုလင်သည်ကား ဘုရား တရား သံဃာ၏ဂုဏ်ကို\myendnote{အကျယ်ရှင်းလင်းချက်။} များစွာသော အကြောင်းဖြင့် ပြော၏၊ ဤသို့လျှင် ထိုဆရာတပည့်နှစ်ဦးတို့သည် အချင်းချင်း တိုက်ရိုက်ဆန့်ကျင်သော\myendnote{another note.} စကားကို ပြောလျက် မြတ်စွာဘုရားနှင့်\marginnotealt{\citemm{ဓမ္မသင်္ဂဏီပါဠိ}[၁၇]} ရဟန်းသံဃာ၏ နောက်မှ နောက်မှ အစဉ်လိုက်ကုန်၏။


% For Printing End Notes(Commentaryand Table of Meanings)
\commentarytable
 % Editorial conventions file

% Main Matter (Arabic numeral pagination)
\mainmatter

\include{text/00}
\chapter{Input Pali Text in Multiple Scripts}

\starthere

\section{Roman Script}
1. Evaṃ me sutaṃ – ekaṃ samayaṃ bhagavā antarā ca rājagahaṃ antarā ca nāḷandaṃ addhānamaggappaṭipanno hoti mahatā bhikkhusaṅghena saddhiṃ pañcamattehi bhikkhusatehi. Suppiyopi kho paribbājako antarā ca rājagahaṃ antarā ca nāḷandaṃ addhānamaggappaṭipanno hoti saddhiṃ antevāsinā brahmadattena māṇavena. Tatra sudaṃ suppiyo paribbājako anekapariyāyena buddhassa avaṇṇaṃ bhāsati, dhammassa avaṇṇaṃ bhāsati, saṅghassa avaṇṇaṃ bhāsati; suppiyassa pana paribbājakassa antevāsī brahmadatto māṇavo anekapariyāyena buddhassa vaṇṇaṃ bhāsati, dhammassa vaṇṇaṃ bhāsati, saṅghassa vaṇṇaṃ bhāsati. Itiha te ubho ācariyantevāsī aññamaññassa ujuvipaccanīkavādā bhagavantaṃ piṭṭhito piṭṭhito anubandhā [anubaddhā (ka. sī. pī.)] honti bhikkhusaṅghañca.

\section{Myanmar Script}
၁။ ဧဝံ မေ သုတံ – ဧကံ သမယံ ဘဂဝါ အန္တရာ စ ရာဇဂဟံ အန္တရာ စ နာဠန္ဒံ အဒ္ဓါနမဂ္ဂပ္ပဋိပန္နော ဟောတိ မဟတာ ဘိက္ခုသင်္ဃေန သဒ္ဓိံ ပဉ္စမတ္တေဟိ ဘိက္ခုသတေဟိ။ သုပ္ပိယောပိ ခေါ ပရိဗ္ဗာဇကော အန္တရာ စ ရာဇဂဟံ အန္တရာ စ နာဠန္ဒံ အဒ္ဓါနမဂ္ဂပ္ပဋိပန္နော ဟောတိ သဒ္ဓိံ အန္တေဝါသိနာ ဗြဟ္မဒတ္တေန မာဏဝေန။ တတြ သုဒံ သုပ္ပိယော ပရိဗ္ဗာဇကော အနေကပရိယာယေန ဗုဒ္ဓဿ အဝဏ္ဏံ ဘာသတိ၊ ဓမ္မဿ အဝဏ္ဏံ ဘာသတိ၊ သင်္ဃဿ အဝဏ္ဏံ ဘာသတိ၊ သုပ္ပိယဿ ပန ပရိဗ္ဗာဇကဿ အန္တေဝါသီ ဗြဟ္မဒတ္တော မာဏဝေါ အနေကပရိယာယေန ဗုဒ္ဓဿ ဝဏ္ဏံ ဘာသတိ၊ ဓမ္မဿ ဝဏ္ဏံ ဘာသတိ၊ သင်္ဃဿ ဝဏ္ဏံ ဘာသတိ။ ဣတိဟ တေ ဥဘော အာစရိယန္တေဝါသီ အညမညဿ ဥဇုဝိပစ္စနီကဝါဒါ ဘဂဝန္တံ ပိဋ္ဌိတော ပိဋ္ဌိတော အနုဗန္ဓာ [အနုဗဒ္ဓါ (က။ သီ။ ပီ။)] ဟောန္တိ ဘိက္ခုသင်္ဃဉ္စ။

\section{Sinhala Script}
1. එවං මෙ සුතං – එකං සමයං භගවා අන්තරා ච රාජගහං අන්තරා ච නාළන්දං අද්ධානමග්ගප්පටිපන්නො හොති මහතා භික්ඛුසඞ්ඝෙන සද්ධිං පඤ්චමත්තෙහි භික්ඛුසතෙහි. සුප්පියොපි ඛො පරිබ්බාජකො අන්තරා ච රාජගහං අන්තරා ච නාළන්දං අද්ධානමග්ගප්පටිපන්නො හොති සද්ධිං අන්තෙවාසිනා බ්රහ්මදත්තෙන මාණවෙන. තත්ර සුදං සුප්පියො පරිබ්බාජකො අනෙකපරියායෙන බුද්ධස්ස අවණ්ණං භාසති, ධම්මස්ස අවණ්ණං භාසති, සඞ්ඝස්ස අවණ්ණං භාසති; සුප්පියස්ස පන පරිබ්බාජකස්ස අන්තෙවාසී බ්රහ්මදත්තො මාණවො අනෙකපරියායෙන බුද්ධස්ස වණ්ණං භාසති, ධම්මස්ස වණ්ණං භාසති, සඞ්ඝස්ස වණ්ණං භාසති. ඉතිහ තෙ උභො ආචරියන්තෙවාසී අඤ්ඤමඤ්ඤස්ස උජුවිපච්චනීකවාදා භගවන්තං පිට්ඨිතො පිට්ඨිතො අනුබන්ධා [අනුබද්ධා (ක. සී. පී.)] හොන්ති භික්ඛුසඞ්ඝඤ්ච.

\section{Devanagari Script}
१. एवं मे सुतं – एकं समयं भगवा अन्तरा च राजगहं अन्तरा च नाळन्दं अद्धानमग्गप्पटिपन्नो होति महता भिक्खुसङ्घेन सद्धिं पञ्चमत्तेहि भिक्खुसतेहि. सुप्पियोपि खो परिब्बाजको अन्तरा च राजगहं अन्तरा च नाळन्दं अद्धानमग्गप्पटिपन्नो होति सद्धिं अन्तेवासिना ब्रह्मदत्तेन माणवेन. तत्र सुदं सुप्पियो परिब्बाजको अनेकपरियायेन बुद्धस्स अवण्णं भासति, धम्मस्स अवण्णं भासति, सङ्घस्स अवण्णं भासति; सुप्पियस्स पन परिब्बाजकस्स अन्तेवासी ब्रह्मदत्तो माणवो अनेकपरियायेन बुद्धस्स वण्णं भासति, धम्मस्स वण्णं भासति, सङ्घस्स वण्णं भासति. इतिह ते उभो आचरियन्तेवासी अञ्ञमञ्ञस्स उजुविपच्चनीकवादा भगवन्तं पिट्ठितो पिट्ठितो अनुबन्धा [अनुबद्धा (क. सी. पी.)] होन्ति भिक्खुसङ्घञ्च.
\chapter{Exposition of Mental Factors (Cetasika)}

\chapter{Practical Writing Example}
\starthere

\section{What is Citta}
You can try applying the commands\footnotealt{You can try applying the commands from the previous chapter to actually write the content of this chapter.} from the previous chapter to actually write the content of this chapter.

You can try applying the commands from the previous chapter\myendnote{You can try applying the commands from the previous chapter to actually write the content of this chapter.\commentary{You can try applying the commands from the previous chapter\footnotealt{\cite{mulapannasa_atthakatha_1}} to actually write the content of this chapter.}} to actually write the content of this chapter.

You can try applying the commands from the previous chapter to actually write the content of this chapter.

You can try applying the commands from the previous chapter to actually write the content of this chapter.

You can try applying the commands from the previous chapter to actually write the content of this chapter.

You can try applying the commands from the previous chapter to actually write the content of this chapter.

You can try applying the commands from the previous chapter to actually write the content of this chapter.

You can try applying the commands from the previous chapter to actually write the content of this chapter.

You can try applying the commands from the previous chapter to actually write the content of this chapter.

You can try applying the commands from the previous chapter to actually write the content of this chapter.

\section{What is Cetasika}
\cetasika{စေတသိက်} is \peyyala{}

% For Printing End Notes(Commentary and Table of Meanings)
\commentarytable% Put at the end of chapter
\include{text/04}
\include{text/05}
\include{text/06}
\include{text/07}
\include{text/08}
\include{text/09}
\include{text/10}
\include{text/11}
\include{text/12}
\include{text/13}
\include{text/14}
\include{text/15}
\include{text/16}
\include{text/17}
\include{text/18}
\include{text/19}
\include{text/20}
\include{text/21}
\include{text/22}
\include{text/23}
\include{text/24}

% Appendix Section (commented, enable as needed)
%\appendix % Start appendices
%\include{appendixA} % Appendix A content
\printunsrtglossary[title=Glossary of Terms, toctitle=Glossary of Terms]

% Back Matter (No page numbering)
\backmatter

% Epilogue Chapter (commented, enable as needed)
%\include{epilogue} % Epilogue content

% Bibliography System
%\nocite{*} % Cite all references (commented)
\printbibliography % Print bibliography (using biblatex)
\markboth{Bibliography}{Bibliography} % Fix running headers
\addcontentsline{toc}{chapter}{Bibliography} % Add to table of contents

% Index System
\printindex % Generate index using makeindex tool

%%%%%%%%%%%%%%%%%%%%%%%%%%%%%%%%%%%%%%%%%%%%%%%%%%%%%%%%%%%%%%%

\end{document}