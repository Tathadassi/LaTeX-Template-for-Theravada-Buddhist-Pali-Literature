% Abhidhammāvatāra Main Document
% Main document for Theravada Buddhist Abhidhamma literature translation and commentary
% MUST be compiled with LuaLaTeX (supports Burmese and Pali script)
%%%%%%%%%%%%%%%%%%%%%%%%%%%%%%%%%%%%%%%%%%%%%%%%%%%%%%%%%%%%%%%

% Version Information
% ======================================================================
% Version: 1.0.0
% Release Date: 2025-09-02
% Author: Venerable Gunodaya Bhikkhu
% Maintainer: Venerable Gunodaya Bhikkhu
% License: Dhamma Dana (Dhamma Gift)
%
% Version History:
% 1.0.0 (2025-09-02) - Initial version release
%   - Completed basic Abhidhamma terminology system
%   - Implemented multilingual typesetting (English-Pali-Burmese)
%   - Established complete annotation system (footnotes, endnotes, margin notes)
%   - Designed professional academic typesetting format
%
% Compilation Requirements:
% - Engine: LuaLaTeX (Required)
% - Fonts: Padauk (Burmese), Times New Roman (Latin)
% - Packages: See abhidhammavatara.cls documentation
%
% Project Structure:
% - text/cover.tex     : Cover design
% - text/000.tex       : Editorial conventions
% - text/01-24.tex     : Chapter contents
% - text/term.tex      : Term definitions
% - references_Manual.bib: Bibliography database
%
% Important Notes:
% 1. MUST be compiled with LuaLaTeX to support Burmese script rendering
% 2. Ensure Padauk font is installed on your system
% 3. Uses biblatex for bibliography, NOT BibTeX
% =====================================================================%

% Document Class Settings
\documentclass[]{Abhidhammāvatāra} % Use the custom class

% Bibliography Database Setup
\addbibresource{references_Manual.bib} % Load bibliography database for biblatex
% Note: This uses biblatex, NOT BibTeX

%%%%%%%%%%%%%%%%%%%%%%%%%%%%%%%%%%%%%%%%%%%%%%%%%%%%%%%%%%%%%%%

% Document Metadata
\title{\textbf{USER MANUAL}} % Pali title
\author{GUṆODAYA BHIKKHU}
\date{} % Publication date left empty

%%%%%%%%%%%%%%%%%%%%%%%%%%%%%%%%%%%%%%%%%%%%%%%%%%%%%%%%%%%%%%%

\begin{document}

% Cover Page
% Abhidhammāvatāra Cover Page

%%%%%%%%%%%%%%%%%%%%%%%%%%%%%%%%%%%%%%%%%
% Cover Page
%%%%%%%%%%%%%%%%%%%%%%%%%%%%%%%%%%%%%%%%%
% Formal Book Title Page
% LaTeX Template
% Version 2.0 (23/7/17)
%
% This template was downloaded from:
% http://www.LaTeXTemplates.com
%
% Original author:
% Peter Wilson (herries.press@earthlink.net) with modifications by:
% Vel (vel@latextemplates.com)
%
% License:
% CC BY-NC-SA 3.0 (http://creativecommons.org/licenses/by-nc-sa/3.0/)
% 
% This template can be used in one of two ways:
%
% 1) Content can be added at the end of this file just before the \end{document}
% to use this title page as the starting point for your document.
%
% 2) Alternatively, if you already have a document which you wish to add this
% title page to, copy everything between the \begin{document} and
% \end{document} and paste it where you would like the title page in your
% document. You will then need to insert the packages and document 
% configurations into your document carefully making sure you are not loading
% the same package twice and that there are no clashes.
%
%%%%%%%%%%%%%%%%%%%%%%%%%%%%%%%%%%%%%%%%%

%----------------------------------------------------------------------------------------
%	PACKAGES AND OTHER DOCUMENT CONFIGURATIONS
%----------------------------------------------------------------------------------------

%\documentclass[a4paper, 11pt, oneside]{book} % A4 paper size, default 11pt font size and oneside for equal margins

% Virtual Publisher
\newcommand{\plogo}{Dhamma Dāna} % Dhamma Gift

% Western font packages - disabled as they cause issues with Pali diacritics
%\usepackage[utf8]{inputenc} % Required for inputting international characters
%\usepackage[T1]{fontenc} % Output font encoding for international characters
%\usepackage{fouriernc} % Use the New Century Schoolbook font

%----------------------------------------------------------------------------------------
%	TITLE PAGE
%----------------------------------------------------------------------------------------

%\begin{document}
\frontmatter
% Set new page layout
\newgeometry{top=3cm, bottom=3cm, outer=3cm, inner=3cm}
\begin{titlepage} % Suppresses headers and footers on the title page

	\centering % Centre everything on the title page
	
	\scshape % Use small caps for all text on the title page
	
	\vspace*{\baselineskip} % White space at the top of the page
	
	%------------------------------------------------
	%	Title
	%------------------------------------------------
	
	\rule{\textwidth}{1.6pt}\vspace*{-\baselineskip}\vspace*{2pt} % Thick horizontal rule
	\rule{\textwidth}{0.4pt} % Thin horizontal rule
	
	\vspace{0.75\baselineskip} % Whitespace above the title
	
	{\LARGE \textbf{ABHIDHAMMĀVATĀRA}\\ \Large{Translation and Commentary}} % Title
	
	\vspace{0.75\baselineskip} % Whitespace below the title
	
	\rule{\textwidth}{0.4pt}\vspace*{-\baselineskip}\vspace{3.2pt} % Thin horizontal rule
	\rule{\textwidth}{1.6pt} % Thick horizontal rule
	
	\vspace{2\baselineskip} % Whitespace after the title block
	
	%------------------------------------------------
	%	Subtitle
	%------------------------------------------------
	
	The Best Vessel for Crossing the Ocean of Abhidhamma % Subtitle or further description
	
	\vspace*{3\baselineskip} % Whitespace under the subtitle
	
	%------------------------------------------------
	%	Editor(s)
	%------------------------------------------------
	
%	Edited By
	
	\vspace{0.5\baselineskip} % Whitespace before the editors
	
	{\scshape\Large \begin{table}[h]
	\centering
	\begin{tabular}{rl}
		Venerable Buddhadatta & Composed \\
%	    Venerable Sumaṅgala & New Explanation   \\
%	    Ancient Masters & Explanatory Notes   \\
		Venerable Gunodaya Bhikkhu  &  Translated and Commentated
	\end{tabular}
%	\caption{}
%	\label{tab:my-table}
\end{table}
} % Editor list
	
	\vspace{0.5\baselineskip} % Whitespace below the editor list
	
%	\textit{The University of California \\ Berkeley} % Editor affiliation
	
	\vfill % Whitespace between editor names and publisher logo
	
	%------------------------------------------------
	%	Publisher
	%------------------------------------------------
	
	\plogo % Publisher logo
	
	\vspace{0.3\baselineskip} % Whitespace under the publisher logo
	
	2025 % Publication year
	
%	{\large publisher} % Publisher

\end{titlepage}

%\end{document}
%----------------------------------------------------------------------------------------
 % Cover design file

% Front Matter (Roman numeral pagination)
\frontmatter

% Title Page
\maketitle % Print title page information
\cleardoublepage % Blank page

% Index Collection Setup (commented, enable as needed)
%\makeindex % Enable index collection

%%%%%%%%%%%%%%%%%%%%%%%%%%%%%%%%%%%%%%%%%%%%%%%%%%%%%%%%%%%%%%%

% Table of Contents System Setup

% Brief Table of Contents (Chapter level only)
\setcounter{tocdepth}{0} % Show only chapter level
\renewcommand{\contentsname}{Brief Contents} % Brief TOC title
\tableofcontents % Print brief table of contents

% Detailed Table of Contents (Full structure)
\renewcommand{\contentsname}{Detailed Contents} % Detailed TOC title

% Note: The following code is commented to allow deeper section levels
% \addtocontents{toc}{\protect\setcounter{tocdepth}{1}}

%%%%%%%%%%%%%%%%%%%%%%%%%%%%%%%%%%%%%%%%%%%%%%%%%%%%%%%%%%%%%%%

% Editorial Conventions Chapter
\chapter{User Manual}

\starthere

\section{Pali Text}
8.
Cittaṃ cetasikaṃ rūpaṃ, nibbānanti niruttaro;
Catudhā desayī dhamme, catusaccappakāsano.

Tattha cittanti visayavijānanaṃ cittaṃ, tassa pana ko vacanattho? Vuccate – sabbasaṅgāhakavasena pana cintetīti cittaṃ, attasantānaṃ vā cinotītipi cittaṃ.

9.
Vicittakaraṇā cittaṃ, attano cittatāya vā;
Paññattiyampi viññāṇe, vicitte cittakammake;
Cittasammuti daṭṭhabbā, viññāṇe idha viññunā.

Taṃ pana sārammaṇato ekavidhaṃ, savipākāvipākato duvidhaṃ. Tattha savipākaṃ nāma kusalākusalaṃ, avipākaṃ abyākataṃ. Kusalākusalābyākatajātibhedato tividhaṃ.

Tattha kusalanti panetassa ko vacanattho?

10.
Kucchitānaṃ salanato, kusānaṃ lavanena vā;
Kusena lātabbattā vā, kusalanti pavuccati.

\section{Burmese Translation}
 ၁။ အကျွန်ုပ်သည် ဤသို့ ကြားနာခဲ့ရပါသည် -
       အခါတစ်ပါး၌ မြတ်စွာဘုရားသည် များစွာသော ငါးရာခန့်မျှသော ရဟန်းသံဃာနှင့်အတူ ရာဇဂြိုဟ် (ပြည်) နှင့် နာဠန္ဒာ (မြို့) အကြား၌ ခရီးရှည်ကြွတော်မူ၏၊ သုပ္ပိယပရိဗိုဇ်သည်လည်း တပည့်ဖြစ်သော ဗြဟ္မဒတ်လုလင်နှင့်အတူ ရာဇဂြိုဟ် (ပြည်) နှင့် နာဠန္ဒာ (မြို့) အကြား၌ပင် ခရီးရှည်သွား၏။
       ထို (ခရီး) ၌ သုပ္ပိယပရိဗိုဇ်သည် ဘုရား တရား သံဃာ၏ အပြစ်ကို များစွာသော အကြောင်းဖြင့် ပြော၏၊ သုပ္ပိယပရိဗိုဇ်၏ တပည့်ဖြစ်သော ဗြဟ္မဒတ်လုလင်သည်ကား ဘုရား တရား သံဃာ၏ဂုဏ်ကို များစွာသော အကြောင်းဖြင့် ပြော၏၊ ဤသို့လျှင် ထိုဆရာတပည့်နှစ်ဦးတို့သည် အချင်းချင်း တိုက်ရိုက်ဆန့်ကျင်သော စကားကို ပြောလျက် မြတ်စွာဘုရားနှင့် ရဟန်းသံဃာ၏ နောက်မှ နောက်မှ အစဉ်လိုက်ကုန်၏။

\section{Academic Annotation System}
\canon{This text is from pāḷi canon.}

\commentary{This text is from aṭṭhakatha.}

\subcommentary{This text is from ṭīkā.}

\othertext{This text is from othertext.}

\section{Grammatical Analysis}
\kitabasic{}{}{}

\kitafull{}{}{}{}

\kitafull*{}{}{}{}

\taddhita{}{}{}

\samasa{}{}{}

\namapada{}{}{}

\akhyatapada{}{}{}{}


\section{Notes System}
\subsection{Inline Notes}
 ၁။ အကျွန်ုပ်သည် ဤသို့ ကြားနာခဲ့ရပါသည် -
       အခါတစ်ပါး၌ မြတ်စွာဘုရားသည်\inlinenote{This is Inline Notes from othertext.} များစွာသော ငါးရာခန့်မျှသော ရဟန်းသံဃာနှင့်အတူ\inlinenote*{This is Inline Notes from aṭṭhakatha.} ရာဇဂြိုဟ် (ပြည်) နှင့် နာဠန္ဒာ (မြို့) အကြား၌ ခရီးရှည်ကြွတော်မူ၏၊ သုပ္ပိယပရိဗိုဇ်သည်လည်း တပည့်ဖြစ်သော ဗြဟ္မဒတ်လုလင်နှင့်အတူ ရာဇဂြိုဟ် (ပြည်) နှင့် နာဠန္ဒာ (မြို့) အကြား၌ပင် ခရီးရှည်သွား၏။
       ထို (ခရီး) ၌ သုပ္ပိယပရိဗိုဇ်သည် ဘုရား တရား သံဃာ၏ အပြစ်ကို များစွာသော အကြောင်း\inlinenote*[]{This is Inline Notes from ṭīkā(ဋီကာ).}ဖြင့် ပြော၏၊ 

\subsection{End Notes}
သုပ္ပိယပရိဗိုဇ်၏ တပည့်ဖြစ်သော ဗြဟ္မဒတ်လုလင်သည်ကား ဘုရား တရား သံဃာ၏ဂုဏ်ကို\myendnote{အကျယ်ရှင်းလင်းချက်။} များစွာသော အကြောင်းဖြင့် ပြော၏၊ ဤသို့လျှင် ထိုဆရာတပည့်နှစ်ဦးတို့သည် အချင်းချင်း တိုက်ရိုက်ဆန့်ကျင်သော စကားကို ပြောလျက် မြတ်စွာဘုရားနှင့် ရဟန်းသံဃာ၏ နောက်မှ နောက်မှ အစဉ်လိုက်ကုန်၏။

\subsection{Inline Notes}


\subsection{Inline Notes}


% For Printing End Notes(Commentaryand Table of Meanings)
\commentarytable % Editorial conventions file

% Main Matter (Arabic numeral pagination)
\mainmatter

\include{text/00}
\chapter{Exposition of Consciousness (Citta)}

\starthere

\section{Consciousness}
8.
Cittaṃ cetasikaṃ rūpaṃ, nibbānanti niruttaro;
Catudhā desayī dhamme, catusaccappakāsano.

Tattha cittanti visayavijānanaṃ cittaṃ, tassa pana ko vacanattho? Vuccate – sabbasaṅgāhakavasena pana cintetīti cittaṃ, attasantānaṃ vā cinotītipi cittaṃ.

9.
Vicittakaraṇā cittaṃ, attano cittatāya vā;
Paññattiyampi viññāṇe, vicitte cittakammake;
Cittasammuti daṭṭhabbā, viññāṇe idha viññunā.

Taṃ pana sārammaṇato ekavidhaṃ, savipākāvipākato duvidhaṃ. Tattha savipākaṃ nāma kusalākusalaṃ, avipākaṃ abyākataṃ. Kusalākusalābyākatajātibhedato tividhaṃ.

Tattha kusalanti panetassa ko vacanattho?

10.
Kucchitānaṃ salanato, kusānaṃ lavanena vā;
Kusena lātabbattā vā, kusalanti pavuccati.

\section{Burmese translation}
စိတ်ဟူသည် \cite{dhammasangani_pali}

\citta{စိတ်} is from \citemm{ဓမ္မသင်္ဂဏီပါဠိ}[၁၇]
\chapter{Exposition of Mental Factors (Cetasika)}

\chapter{Practical Writing Example}
\starthere

\section{What is Citta}
You can try applying the commands\footnotealt{You can try applying the commands from the previous chapter to actually write the content of this chapter.} from the previous chapter to actually write the content of this chapter.

You can try applying the commands from the previous chapter\myendnote{You can try applying the commands from the previous chapter to actually write the content of this chapter.\commentary{You can try applying the commands from the previous chapter\footnotealt{\cite{mulapannasa_atthakatha_1}} to actually write the content of this chapter.}} to actually write the content of this chapter.

You can try applying the commands from the previous chapter to actually write the content of this chapter.

You can try applying the commands from the previous chapter to actually write the content of this chapter.

You can try applying the commands from the previous chapter to actually write the content of this chapter.

You can try applying the commands from the previous chapter to actually write the content of this chapter.

You can try applying the commands from the previous chapter to actually write the content of this chapter.

You can try applying the commands from the previous chapter to actually write the content of this chapter.

You can try applying the commands from the previous chapter to actually write the content of this chapter.

You can try applying the commands from the previous chapter to actually write the content of this chapter.

\section{What is Cetasika}
\cetasika{စေတသိက်} is \peyyala{}

% For Printing End Notes(Commentary and Table of Meanings)
\commentarytable% Put at the end of chapter
\include{text/04}
\include{text/05}
\include{text/06}
\include{text/07}
\include{text/08}
\include{text/09}
\include{text/10}
\include{text/11}
\include{text/12}
\include{text/13}
\include{text/14}
\include{text/15}
\include{text/16}
\include{text/17}
\include{text/18}
\include{text/19}
\include{text/20}
\include{text/21}
\include{text/22}
\include{text/23}
\include{text/24}

% Appendix Section (commented, enable as needed)
%\appendix % Start appendices
%\include{appendixA} % Appendix A content
\printunsrtglossary[title=Glossary of Terms, toctitle=Glossary of Terms]

% Back Matter (No page numbering)
\backmatter

% Epilogue Chapter (commented, enable as needed)
%\include{epilogue} % Epilogue content

% Bibliography System
%\nocite{*} % Cite all references (commented)
\printbibliography % Print bibliography (using biblatex)
\markboth{Bibliography}{Bibliography} % Fix running headers
\addcontentsline{toc}{chapter}{Bibliography} % Add to table of contents

% Index System
\printindex % Generate index using makeindex tool

%%%%%%%%%%%%%%%%%%%%%%%%%%%%%%%%%%%%%%%%%%%%%%%%%%%%%%%%%%%%%%%

\end{document}