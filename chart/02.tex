% For Printing End Notes(Commentary and Table of Meanings)
\commentarytable

\begin{table}[h]
	\centering
 \caption{အနုဗန္ဓကဝီထိ} \label{}
%\vspace{1ex}
 \begin{subtable}[b]{1.0\linewidth}
	\centering
     \caption{ဖြစ်ဆဲသိ စိတ်အစဉ်} \label{1}
	\begin{tabular}{ccccc|cccccccccccccc|cc}
		ဘ & & တီ & န & ဒ & ပ & စ & သံ & ဏ & ဝု & ဇ & ဇ & ဇ & ဇ & ဇ & ဇ & ဇ & ရုံ & ရုံ &  & ဘ \\
  \cmidrule{6-19}
  &&&&& \multicolumn{14}{c|}{ပစ္စုပ္ပန် ရူပါရုံ (ပရမတ်)} && \\
	\end{tabular}
 \end{subtable}
\vspace{2ex}

 \begin{subtable}[b]{1.0\linewidth}
	\centering
     \caption{ဖြစ်ပြီးသိ စိတ်အစဉ်} \label{2}
	\begin{tabular}{cccc|cccccccccc|cc}
		ဘ &  & န & ဒ & မ & ဇ & ဇ & ဇ & ဇ & ဇ & ဇ & ဇ & ရုံ & ရုံ &  & ဘ \\
    \cmidrule{5-14}
  &&&& \multicolumn{10}{c|}{အတိတ် ရူပါရုံ (ပရမတ်)} && \\
	\end{tabular}
 \end{subtable}
\vspace{2ex}

 \begin{subtable}[b]{1.0\linewidth}
	\centering
%     \caption{အ​ပေါင်းသိ စိတ်အစဉ်\jiaozhu{此心路的所缘到底是ပရမတ်,还是ပညတ်,有不同观点。ဂန္ဓာရုံ认为是ပရမတ်,လယ်တီ认为是ပညတ်。}} \label{3}
	\begin{tabular}{cccc|cccccccccc|cc}
		ဘ &  & န & ဒ & မ & ဇ & ဇ & ဇ & ဇ & ဇ & ဇ & ဇ & ရုံ & ရုံ &  & ဘ \\
    \cmidrule{5-14}
  &&&& \multicolumn{10}{c|}{အတိတ် ရူပါရုံ အ​ပေါင်း (ပရမတ်)} && \\
	\end{tabular}
 \end{subtable}
\vspace{2ex}

 \begin{subtable}[b]{1.0\linewidth}
	\centering
     \caption{ဒြပ်သိ စိတ်အစဉ်} \label{4}
	\begin{tabular}{cccc|cccccccc|cccc}
		ဘ &  & န & ဒ & မ & ဇ & ဇ & ဇ & ဇ & ဇ & ဇ & ဇ &  &  &  & ဘ \\
  \cmidrule{5-12}
  &&&& \multicolumn{8}{c|}{ပုံရိပ် (ပညတ်)} &&&& \\
	\end{tabular}
 \end{subtable}
\vspace{2ex}

 \begin{subtable}[b]{1.0\linewidth}
	\centering
     \caption{အမည်သိ စိတ်အစဉ်} \label{5}
	\begin{tabular}{cccc|cccccccc|cccc}
		ဘ &  & န & ဒ & မ & ဇ & ဇ & ဇ & ဇ & ဇ & ဇ & ဇ &  &  &  & ဘ \\
  \cmidrule{5-12}
  &&&& \multicolumn{8}{c|}{အမည် (ပညတ်)} &&&& \\
	\end{tabular}
 \end{subtable}
 
\end{table}