\chapter{Translation and Commentary}
\starthere

\section{Burmese Translation}
၁။ အကျွန်ုပ်သည် ဤသို့ ကြားနာခဲ့ရပါသည် - အခါတစ်ပါး၌ မြတ်စွာဘုရားသည် များစွာသော ငါးရာခန့်မျှသော ရဟန်းသံဃာနှင့်အတူ ရာဇဂြိုဟ် (ပြည်) နှင့် နာဠန္ဒာ (မြို့) အကြား၌ ခရီးရှည်ကြွတော်မူ၏၊ သုပ္ပိယပရိဗိုဇ်သည်လည်း တပည့်ဖြစ်သော ဗြဟ္မဒတ်လုလင်နှင့်အတူ ရာဇဂြိုဟ် (ပြည်) နှင့် နာဠန္ဒာ (မြို့) အကြား၌ပင် ခရီးရှည်သွား၏။ ထို (ခရီး) ၌ သုပ္ပိယပရိဗိုဇ်သည် ဘုရား တရား သံဃာ၏ အပြစ်ကို များစွာသော အကြောင်းဖြင့် ပြော၏၊ သုပ္ပိယပရိဗိုဇ်၏ တပည့်ဖြစ်သော ဗြဟ္မဒတ်လုလင်သည်ကား ဘုရား တရား သံဃာ၏ဂုဏ်ကို များစွာသော အကြောင်းဖြင့် ပြော၏၊ ဤသို့လျှင် ထိုဆရာတပည့်နှစ်ဦးတို့သည် အချင်းချင်း တိုက်ရိုက်ဆန့်ကျင်သော စကားကို ပြောလျက် မြတ်စွာဘုရားနှင့် ရဟန်းသံဃာ၏ နောက်မှ နောက်မှ အစဉ်လိုက်ကုန်၏။

\section{Chinese Translation}
1.1 如是我闻。一时,世尊与五百大比丘僧在王舍城与那烂陀之间的旅途上行进。有游方者善念与青年弟子梵赐也在王舍城与那烂陀之间的旅途上行进。那时,游方者善念以多种方式毁谤佛、毁谤法、毁谤僧,而游方者善念的青年弟子梵赐却以多种方式称赞佛、称赞法、称赞僧。如此,师徒二人彼此言辞直接对立,跟随于世尊与比丘僧之后。

\section{English Translation}
Thus I have heard: One time, the Buddha was at the Flowering Grove Hut in Jeta’s Grove of Śrāvastī. He was accompanied by a large assembly of 1,250 monks.

It was then that the monks gathered in the Flowering Grove Hall after soliciting alms. They engaged in a discussion with each other: “Venerable monks, only the unsurpassed sage is so extraordinary! His miraculous powers are far-reaching, and his authority is tremendous. He has come to know the countless buddhas of the past who have entered nirvāṇa, broken the bonds, and eliminated idle speculation.”

\section{Academic Annotation System}
\canon{This text is from pāḷi canon.}

\commentary{This text is from aṭṭhakatha.}

\subcommentary{This text is from ṭīkā.}

\othertext{This text is from othertext.}

\section{Grammatical Analysis}
\subsection{ကိတ်}
\kitabasic{လိင်္ဂ}{ဓာတု}{ပစ္စယ}

\kitabasic{ဗုဒ္ဓ}[မြတ်စွာဘုရား]{ဗုဓ}[သိ]{တ}[တတ်]

\kitafull{လိင်္ဂ}{လိင်္ဂ}{ဓာတု}{ပစ္စယ}

\kitafull{ဝိစာရ}{ဝိ}{စရ}{ဏ}

\kitafull*{လိင်္ဂ}{ဓာတု}{ပစ္စယ}{ပစ္စယ}

\kitafull*{ဗုဒ္ဓ}{ဗုဓ}{ဏေ}[စေ]{တ}

\subsection{တဒ္ဓိတ်}
\taddhita{လိင်္ဂ}{လိင်္ဂ}{ပစ္စယ}

\subsection{သမာသ်}
\samasa{လိင်္ဂ}{လိင်္ဂ}{လိင်္ဂ}

\subsection{နာမ်}
\namapada{နာမ}{လိင်္ဂ}{ဝိဘတ္တိ}

\namapada{buddho}{buddha}{si}[သည်]

\subsection{အာချာတ်}
\akhyatapada{အာချာတ}{ဓာတု}{ပစ္စယ}{ဝိဘတ္တိ}

\akhyatapada{gacchati}{gamu}{a}{ti}


\section{Notes System}
\subsection{Inline Notes}
၁။ အကျွန်ုပ်သည် ဤသို့ ကြားနာခဲ့ရပါသည် - အခါတစ်ပါး၌ မြတ်စွာဘုရားသည်\inlinenote{This is Inline Notes from othertext.} များစွာသော ငါးရာခန့်မျှသော ရဟန်းသံဃာနှင့်အတူ\inlinenote*{This is Inline Notes from aṭṭhakatha.} ရာဇဂြိုဟ် (ပြည်) နှင့် နာဠန္ဒာ (မြို့) အကြား၌ ခရီးရှည်ကြွတော်မူ၏၊ သုပ္ပိယပရိဗိုဇ်သည်လည်း တပည့်ဖြစ်သော ဗြဟ္မဒတ်လုလင်နှင့်အတူ ရာဇဂြိုဟ် (ပြည်) နှင့် နာဠန္ဒာ (မြို့) အကြား၌ပင် ခရီးရှည်သွား၏။ ထို (ခရီး) ၌ သုပ္ပိယပရိဗိုဇ်သည် ဘုရား တရား သံဃာ၏ အပြစ်ကို များစွာသော အကြောင်း\inlinenote*[]{This is Inline Notes from ṭīkā(ဋီကာ).}ဖြင့် ပြော၏၊ 

\subsection{End Notes}
သုပ္ပိယပရိဗိုဇ်၏ တပည့်ဖြစ်သော ဗြဟ္မဒတ်လုလင်သည်ကား ဘုရား တရား သံဃာ၏ဂုဏ်ကို\myendnote{အကျယ်ရှင်းလင်းချက်။ \commentary{This text is from aṭṭhakatha.}\footnotealt{\cite{atthasalini}}} များစွာသော အကြောင်းဖြင့် ပြော၏၊ ဤသို့လျှင် ထိုဆရာ တပည့်နှစ်ဦးတို့သည် အချင်းချင်း တိုက်ရိုက်ဆန့်ကျင်သော\myendnote{another note.} စကားကို ပြောလျက် မြတ်စွာဘုရားနှင့် ရဟန်းသံဃာ၏ နောက်မှ နောက်မှ အစဉ်လိုက်ကုန်၏။

\subsection{Footnote}
Vicittakaraṇā cittaṃ, attano\footnotealt{This is footnote.} cittatāya vā;
Paññattiyampi viññāṇe, vicitte cittakammake;
Cittasammuti daṭṭhabbā, viññāṇe idha viññunā.


\subsection{Margin Notes}
သုပ္ပိယပရိဗိုဇ်၏ တပည့်ဖြစ်သော\marginnotealt{p.158} ဗြဟ္မဒတ်လုလင်သည်ကား ဘုရား တရား သံဃာ၏ဂုဏ်ကို\myendnote{အကျယ်ရှင်းလင်းချက်။} များစွာသော အကြောင်း\footnotealt{\citemm{ဓမ္မသင်္ဂဏီပါဠိ}[၁၂၃]}ဖြင့် ပြော၏၊ ဤသို့လျှင် ထိုဆရာတပည့်နှစ်ဦးတို့သည် အချင်းချင်း တိုက်ရိုက်ဆန့်ကျင်သော\myendnote{another endnote.} စကားကို ပြောလျက် မြတ်စွာဘုရားနှင့်\marginnotealt{\citemm{ဓမ္မသင်္ဂဏီပါဠိ}[၁၇]} ရဟန်းသံဃာ၏ နောက်မှ နောက်မှ အစဉ်လိုက်ကုန်၏။

\section{Terminology Management System}
\subsection{Unmature Translation}
If you haven't decided how to translate this term yet, you can briefly pass over it with \termdisplay{Original Text}. Once the translation is confirmed, add the term in text/term.tex, and it will be automatically replaced.

\termdisplay{စိတ္တံ}% The display effect is the same as \citta{စိတ်}, which is the result of automatic replacement.

\termdisplay{စိတ္တံ}[Mind]

\termdisplay*{စိတ္တံ}

\termdisplay*{စိတ္တံ}[Mind]

\termdisplay{ဘဂဝါ}

\termdisplay{ဘဂဝါ}[世尊]

\termdisplay*{ဘဂဝါ}

\termdisplay*{ဘဂဝါ}[世尊]

\subsection{Mature Translation}
If there is already a mature and widely accepted translation for this term, it is recommended to directly use the respective specific command.

\citta{စိတ်}

\citta*{စိတ်}


% For Printing End Notes(Commentary and Table of Meanings)
% For Printing End Notes(Commentary and Table of Meanings)
\commentarytable

\begin{table}[h]
	\centering
 \caption{အနုဗန္ဓကဝီထိ} \label{}
%\vspace{1ex}
 \begin{subtable}[b]{1.0\linewidth}
	\centering
     \caption{ဖြစ်ဆဲသိ စိတ်အစဉ်} \label{1}
	\begin{tabular}{ccccc|cccccccccccccc|cc}
		ဘ & & တီ & န & ဒ & ပ & စ & သံ & ဏ & ဝု & ဇ & ဇ & ဇ & ဇ & ဇ & ဇ & ဇ & ရုံ & ရုံ &  & ဘ \\
  \cmidrule{6-19}
  &&&&& \multicolumn{14}{c|}{ပစ္စုပ္ပန် ရူပါရုံ (ပရမတ်)} && \\
	\end{tabular}
 \end{subtable}
\vspace{2ex}

 \begin{subtable}[b]{1.0\linewidth}
	\centering
     \caption{ဖြစ်ပြီးသိ စိတ်အစဉ်} \label{2}
	\begin{tabular}{cccc|cccccccccc|cc}
		ဘ &  & န & ဒ & မ & ဇ & ဇ & ဇ & ဇ & ဇ & ဇ & ဇ & ရုံ & ရုံ &  & ဘ \\
    \cmidrule{5-14}
  &&&& \multicolumn{10}{c|}{အတိတ် ရူပါရုံ (ပရမတ်)} && \\
	\end{tabular}
 \end{subtable}
\vspace{2ex}

 \begin{subtable}[b]{1.0\linewidth}
	\centering
%     \caption{အ​ပေါင်းသိ စိတ်အစဉ်\jiaozhu{此心路的所缘到底是ပရမတ်,还是ပညတ်,有不同观点。ဂန္ဓာရုံ认为是ပရမတ်,လယ်တီ认为是ပညတ်。}} \label{3}
	\begin{tabular}{cccc|cccccccccc|cc}
		ဘ &  & န & ဒ & မ & ဇ & ဇ & ဇ & ဇ & ဇ & ဇ & ဇ & ရုံ & ရုံ &  & ဘ \\
    \cmidrule{5-14}
  &&&& \multicolumn{10}{c|}{အတိတ် ရူပါရုံ အ​ပေါင်း (ပရမတ်)} && \\
	\end{tabular}
 \end{subtable}
\vspace{2ex}

 \begin{subtable}[b]{1.0\linewidth}
	\centering
     \caption{ဒြပ်သိ စိတ်အစဉ်} \label{4}
	\begin{tabular}{cccc|cccccccc|cccc}
		ဘ &  & န & ဒ & မ & ဇ & ဇ & ဇ & ဇ & ဇ & ဇ & ဇ &  &  &  & ဘ \\
  \cmidrule{5-12}
  &&&& \multicolumn{8}{c|}{ပုံရိပ် (ပညတ်)} &&&& \\
	\end{tabular}
 \end{subtable}
\vspace{2ex}

 \begin{subtable}[b]{1.0\linewidth}
	\centering
     \caption{အမည်သိ စိတ်အစဉ်} \label{5}
	\begin{tabular}{cccc|cccccccc|cccc}
		ဘ &  & န & ဒ & မ & ဇ & ဇ & ဇ & ဇ & ဇ & ဇ & ဇ &  &  &  & ဘ \\
  \cmidrule{5-12}
  &&&& \multicolumn{8}{c|}{အမည် (ပညတ်)} &&&& \\
	\end{tabular}
 \end{subtable}
 
\end{table}