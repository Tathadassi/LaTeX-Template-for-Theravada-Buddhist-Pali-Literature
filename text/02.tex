\chapter{Translation in Different Languages}

\section{Burmese Translation}
 ၁။ အကျွန်ုပ်သည် ဤသို့ ကြားနာခဲ့ရပါသည် -
       အခါတစ်ပါး၌ မြတ်စွာဘုရားသည် များစွာသော ငါးရာခန့်မျှသော ရဟန်းသံဃာနှင့်အတူ ရာဇဂြိုဟ် (ပြည်) နှင့် နာဠန္ဒာ (မြို့) အကြား၌ ခရီးရှည်ကြွတော်မူ၏၊ သုပ္ပိယပရိဗိုဇ်သည်လည်း တပည့်ဖြစ်သော ဗြဟ္မဒတ်လုလင်နှင့်အတူ ရာဇဂြိုဟ် (ပြည်) နှင့် နာဠန္ဒာ (မြို့) အကြား၌ပင် ခရီးရှည်သွား၏။
       ထို (ခရီး) ၌ သုပ္ပိယပရိဗိုဇ်သည် ဘုရား တရား သံဃာ၏ အပြစ်ကို များစွာသော အကြောင်းဖြင့် ပြော၏၊ သုပ္ပိယပရိဗိုဇ်၏ တပည့်ဖြစ်သော ဗြဟ္မဒတ်လုလင်သည်ကား ဘုရား တရား သံဃာ၏ဂုဏ်ကို များစွာသော အကြောင်းဖြင့် ပြော၏၊ ဤသို့လျှင် ထိုဆရာတပည့်နှစ်ဦးတို့သည် အချင်းချင်း တိုက်ရိုက်ဆန့်ကျင်သော စကားကို ပြောလျက် မြတ်စွာဘုရားနှင့် ရဟန်းသံဃာ၏ နောက်မှ နောက်မှ အစဉ်လိုက်ကုန်၏။

\section{Chinese Translation}
1.1 如是我闻。一时,世尊与五百大比丘僧在王舍城与那烂陀之间的旅途上行进。有游方者善念与青年弟子梵赐也在王舍城与那烂陀之间的旅途上行进。那时,游方者善念以多种方式毁谤佛、毁谤法、毁谤僧,而游方者善念的青年弟子梵赐却以多种方式称赞佛、称赞法、称赞僧。如此,师徒二人彼此言辞直接对立,跟随于世尊与比丘僧之后。

\section{Academic Annotation System}
\canon{This text is from pāḷi canon.}

\commentary{This text is from aṭṭhakatha.}

\subcommentary{This text is from ṭīkā.}

\othertext{This text is from othertext.}

\section{Grammatical Analysis}
\subsection{ကိတ်}
\kitabasic{}{}{}

\kitabasic{ဗုဒ္ဓ}[မြတ်စွာဘုရား]{ဗုဓ}[သိ]{တ}[တတ်]

\kitafull{}{}{}{}

\kitafull{ဝိစာရ}{ဝိ}{စရ}{ဏ}

\kitafull*{}{}{}{}

\kitafull*{ဗုဒ္ဓ}{ဗုဓ}{ဏေ}[စေ]{တ}

\subsection{တဒ္ဓိတ်}
\taddhita{}{}{}

\subsection{သမာသ်}
\samasa{}{}{}

\subsection{Noun}
\namapada{buddho}{buddha}{si}

\subsection{Verb}
\akhyatapada{gacchati}{gamu}{a}{ti}


\section{Notes System}
\subsection{Inline Notes}
 ၁။ အကျွန်ုပ်သည် ဤသို့ ကြားနာခဲ့ရပါသည် -
       အခါတစ်ပါး၌ မြတ်စွာဘုရားသည်\inlinenote{This is Inline Notes from othertext.} များစွာသော ငါးရာခန့်မျှသော ရဟန်းသံဃာနှင့်အတူ\inlinenote*{This is Inline Notes from aṭṭhakatha.} ရာဇဂြိုဟ် (ပြည်) နှင့် နာဠန္ဒာ (မြို့) အကြား၌ ခရီးရှည်ကြွတော်မူ၏၊ သုပ္ပိယပရိဗိုဇ်သည်လည်း တပည့်ဖြစ်သော ဗြဟ္မဒတ်လုလင်နှင့်အတူ ရာဇဂြိုဟ် (ပြည်) နှင့် နာဠန္ဒာ (မြို့) အကြား၌ပင် ခရီးရှည်သွား၏။
       ထို (ခရီး) ၌ သုပ္ပိယပရိဗိုဇ်သည် ဘုရား တရား သံဃာ၏ အပြစ်ကို များစွာသော အကြောင်း\inlinenote*[]{This is Inline Notes from ṭīkā(ဋီကာ).}ဖြင့် ပြော၏၊ 

\subsection{End Notes}
သုပ္ပိယပရိဗိုဇ်၏ တပည့်ဖြစ်သော ဗြဟ္မဒတ်လုလင်သည်ကား ဘုရား တရား သံဃာ၏ဂုဏ်ကို\myendnote{အကျယ်ရှင်းလင်းချက်။ \commentary{This text is from aṭṭhakatha.}\footnotealt{\cite{atthasalini}}} များစွာသော အကြောင်းဖြင့် ပြော၏၊ ဤသို့လျှင် ထိုဆရာတပည့်နှစ်ဦးတို့သည် အချင်းချင်း တိုက်ရိုက်ဆန့်ကျင်သော\myendnote{another note.} စကားကို ပြောလျက် မြတ်စွာဘုရားနှင့် ရဟန်းသံဃာ၏ နောက်မှ နောက်မှ အစဉ်လိုက်ကုန်၏။

\subsection{Footnote}
Vicittakaraṇā cittaṃ, attano\footnotealt{This is footnote.} cittatāya vā;
Paññattiyampi viññāṇe, vicitte cittakammake;
Cittasammuti daṭṭhabbā, viññāṇe idha viññunā.


\subsection{Margin Notes}
သုပ္ပိယပရိဗိုဇ်၏ တပည့်ဖြစ်သော\marginnotealt{p.158} ဗြဟ္မဒတ်လုလင်သည်ကား ဘုရား တရား သံဃာ၏ဂုဏ်ကို\myendnote{အကျယ်ရှင်းလင်းချက်။} များစွာသော အကြောင်းဖြင့် ပြော၏၊ ဤသို့လျှင် ထိုဆရာတပည့်နှစ်ဦးတို့သည် အချင်းချင်း တိုက်ရိုက်ဆန့်ကျင်သော\myendnote{another note.} စကားကို ပြောလျက် မြတ်စွာဘုရားနှင့်\marginnotealt{\citemm{ဓမ္မသင်္ဂဏီပါဠိ}[၁၇]} ရဟန်းသံဃာ၏ နောက်မှ နောက်မှ အစဉ်လိုက်ကုန်၏။

\section{Terminology Management System}
\termdisplay{စိတ္တံ}

\termdisplay*{စိတ္တံ}

\citta{စိတ်}

\citta*{စိတ်}


% For Printing End Notes(Commentary and Table of Meanings)
\commentarytable
