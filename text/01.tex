\chapter{Input Pali Text in Multiple Scripts}

\starthere

\section{Roman Script}
1. Evaṃ me sutaṃ – ekaṃ samayaṃ bhagavā antarā ca rājagahaṃ antarā ca nāḷandaṃ addhānamaggappaṭipanno hoti mahatā bhikkhusaṅghena saddhiṃ pañcamattehi bhikkhusatehi. Suppiyopi kho paribbājako antarā ca rājagahaṃ antarā ca nāḷandaṃ addhānamaggappaṭipanno hoti saddhiṃ antevāsinā brahmadattena māṇavena. Tatra sudaṃ suppiyo paribbājako anekapariyāyena buddhassa avaṇṇaṃ bhāsati, dhammassa avaṇṇaṃ bhāsati, saṅghassa avaṇṇaṃ bhāsati; suppiyassa pana paribbājakassa antevāsī brahmadatto māṇavo anekapariyāyena buddhassa vaṇṇaṃ bhāsati, dhammassa vaṇṇaṃ bhāsati, saṅghassa vaṇṇaṃ bhāsati. Itiha te ubho ācariyantevāsī aññamaññassa ujuvipaccanīkavādā bhagavantaṃ piṭṭhito piṭṭhito anubandhā [anubaddhā (ka. sī. pī.)] honti bhikkhusaṅghañca.

\section{Myanmar Script}
၁။ ဧဝံ မေ သုတံ – ဧကံ သမယံ ဘဂဝါ အန္တရာ စ ရာဇဂဟံ အန္တရာ စ နာဠန္ဒံ အဒ္ဓါနမဂ္ဂပ္ပဋိပန္နော ဟောတိ မဟတာ ဘိက္ခုသင်္ဃေန သဒ္ဓိံ ပဉ္စမတ္တေဟိ ဘိက္ခုသတေဟိ။ သုပ္ပိယောပိ ခေါ ပရိဗ္ဗာဇကော အန္တရာ စ ရာဇဂဟံ အန္တရာ စ နာဠန္ဒံ အဒ္ဓါနမဂ္ဂပ္ပဋိပန္နော ဟောတိ သဒ္ဓိံ အန္တေဝါသိနာ ဗြဟ္မဒတ္တေန မာဏဝေန။ တတြ သုဒံ သုပ္ပိယော ပရိဗ္ဗာဇကော အနေကပရိယာယေန ဗုဒ္ဓဿ အဝဏ္ဏံ ဘာသတိ၊ ဓမ္မဿ အဝဏ္ဏံ ဘာသတိ၊ သင်္ဃဿ အဝဏ္ဏံ ဘာသတိ၊ သုပ္ပိယဿ ပန ပရိဗ္ဗာဇကဿ အန္တေဝါသီ ဗြဟ္မဒတ္တော မာဏဝေါ အနေကပရိယာယေန ဗုဒ္ဓဿ ဝဏ္ဏံ ဘာသတိ၊ ဓမ္မဿ ဝဏ္ဏံ ဘာသတိ၊ သင်္ဃဿ ဝဏ္ဏံ ဘာသတိ။ ဣတိဟ တေ ဥဘော အာစရိယန္တေဝါသီ အညမညဿ ဥဇုဝိပစ္စနီကဝါဒါ ဘဂဝန္တံ ပိဋ္ဌိတော ပိဋ္ဌိတော အနုဗန္ဓာ [အနုဗဒ္ဓါ (က။ သီ။ ပီ။)] ဟောန္တိ ဘိက္ခုသင်္ဃဉ္စ။

\section{Sinhala Script}
1. එවං මෙ සුතං – එකං සමයං භගවා අන්තරා ච රාජගහං අන්තරා ච නාළන්දං අද්ධානමග්ගප්පටිපන්නො හොති මහතා භික්ඛුසඞ්ඝෙන සද්ධිං පඤ්චමත්තෙහි භික්ඛුසතෙහි. සුප්පියොපි ඛො පරිබ්බාජකො අන්තරා ච රාජගහං අන්තරා ච නාළන්දං අද්ධානමග්ගප්පටිපන්නො හොති සද්ධිං අන්තෙවාසිනා බ්රහ්මදත්තෙන මාණවෙන. තත්ර සුදං සුප්පියො පරිබ්බාජකො අනෙකපරියායෙන බුද්ධස්ස අවණ්ණං භාසති, ධම්මස්ස අවණ්ණං භාසති, සඞ්ඝස්ස අවණ්ණං භාසති; සුප්පියස්ස පන පරිබ්බාජකස්ස අන්තෙවාසී බ්රහ්මදත්තො මාණවො අනෙකපරියායෙන බුද්ධස්ස වණ්ණං භාසති, ධම්මස්ස වණ්ණං භාසති, සඞ්ඝස්ස වණ්ණං භාසති. ඉතිහ තෙ උභො ආචරියන්තෙවාසී අඤ්ඤමඤ්ඤස්ස උජුවිපච්චනීකවාදා භගවන්තං පිට්ඨිතො පිට්ඨිතො අනුබන්ධා [අනුබද්ධා (ක. සී. පී.)] හොන්ති භික්ඛුසඞ්ඝඤ්ච.

\section{Devanagari Script}
१. एवं मे सुतं – एकं समयं भगवा अन्तरा च राजगहं अन्तरा च नाळन्दं अद्धानमग्गप्पटिपन्नो होति महता भिक्खुसङ्घेन सद्धिं पञ्चमत्तेहि भिक्खुसतेहि. सुप्पियोपि खो परिब्बाजको अन्तरा च राजगहं अन्तरा च नाळन्दं अद्धानमग्गप्पटिपन्नो होति सद्धिं अन्तेवासिना ब्रह्मदत्तेन माणवेन. तत्र सुदं सुप्पियो परिब्बाजको अनेकपरियायेन बुद्धस्स अवण्णं भासति, धम्मस्स अवण्णं भासति, सङ्घस्स अवण्णं भासति; सुप्पियस्स पन परिब्बाजकस्स अन्तेवासी ब्रह्मदत्तो माणवो अनेकपरियायेन बुद्धस्स वण्णं भासति, धम्मस्स वण्णं भासति, सङ्घस्स वण्णं भासति. इतिह ते उभो आचरियन्तेवासी अञ्ञमञ्ञस्स उजुविपच्चनीकवादा भगवन्तं पिट्ठितो पिट्ठितो अनुबन्धा [अनुबद्धा (क. सी. पी.)] होन्ति भिक्खुसङ्घञ्च.