\chapter{Exposition of Consciousness (Citta)}

\starthere

\section{Consciousness}
8.
Cittaṃ cetasikaṃ rūpaṃ, nibbānanti niruttaro;
Catudhā desayī dhamme, catusaccappakāsano.

Tattha cittanti visayavijānanaṃ cittaṃ, tassa pana ko vacanattho? Vuccate – sabbasaṅgāhakavasena pana cintetīti cittaṃ, attasantānaṃ vā cinotītipi cittaṃ.

9.
Vicittakaraṇā cittaṃ, attano cittatāya vā;
Paññattiyampi viññāṇe, vicitte cittakammake;
Cittasammuti daṭṭhabbā, viññāṇe idha viññunā.

Taṃ pana sārammaṇato ekavidhaṃ, savipākāvipākato duvidhaṃ. Tattha savipākaṃ nāma kusalākusalaṃ, avipākaṃ abyākataṃ. Kusalākusalābyākatajātibhedato tividhaṃ.

Tattha kusalanti panetassa ko vacanattho?

10.
Kucchitānaṃ salanato, kusānaṃ lavanena vā;
Kusena lātabbattā vā, kusalanti pavuccati.

\section{Burmese translation}
စိတ်ဟူသည် \cite{dhammasangani_pali}

\citta{စိတ်} is from \citemm{ဓမ္မသင်္ဂဏီပါဠိ}[၁၇]

\canon{this}

\commentary{this}

\subcommentary{text}

\othertext{text}
