\chapter{Practical Writing Example}
\starthere

\section{What is Citta}
You can try applying the commands\footnotealt{You can try applying the commands from the previous chapter to actually write the content of this chapter.} from the previous chapter to actually write the content of this chapter.

You can try applying the commands from the previous chapter\myendnote{You can try applying the commands from the previous chapter to actually write the content of this chapter.\commentary{You can try applying the commands from the previous chapter\footnotealt{\cite{mulapannasa_atthakatha_1}} to actually write the content of this chapter.}} to actually write the content of this chapter.

You can try applying the commands from the previous chapter to actually write the content of this chapter.

You can try applying the commands from the previous chapter to actually write the content of this chapter.

You can try applying the commands from the previous chapter to actually write the content of this chapter.

You can try applying the commands from the previous chapter to actually write the content of this chapter.

You can try applying the commands from the previous chapter to actually write the content of this chapter.

You can try applying the commands from the previous chapter to actually write the content of this chapter.

You can try applying the commands from the previous chapter to actually write the content of this chapter.

You can try applying the commands from the previous chapter to actually write the content of this chapter.

\section{What is Cetasika}
\cetasika{စေတသိက်} is \peyyala{}

% For Printing End Notes(Commentary and Table of Meanings)
\commentarytable% Put at the end of chapter